\chapter{Adjectives and quantifiers}
\label{chap:6}

This chapter is primarily about descriptive adjectives and their properties, but some other classes with close affinities to descriptive adjectives, particularly those used in quantifying noun heads, are included and discussed briefly. This is also the rationale for numerals being presented in this chapter. Additionally, there is some focus on semantic aspects of the adjective category and its status vis-à-vis nouns and verbs.


\section{The adjective and its properties}
\label{sec:6-1}


A core group of descriptive adjectives in Palula forms a~lexical category clearly differentiated semantically, morphologically and syntactically from nouns as well as from verbs. It seems, however, size"=wise to be a~rather limited class when compared with nouns and verbs and also less uniform as a~whole. To what extent adjectives should be considered an~open word class will be discussed below. 



Among the adjectives there are those that display a~close affinity with nouns and another group that shares characteristics with some verb forms. A considerable overlap between adjectives and nouns, on the one hand, and between adjectives and participles, on the other, was a~feature also of OIA \citep[322, 967]{whitney1960}. There is also a~group of words that can be used either as adjectives or as adverbs. There are also a~number of properties that tend to be coded as adjectives in languages with large inventories, such as English and Japanese \citep[60]{pustet2006}, that are either verbs or nouns in Palula. Those groups of potential adjectives roughly correspond to the two ``swing"=categories'' suggested by \citet[321]{givon1979}, based on their respective time stability as compared to the most typical members of the class.



Before dealing in more depth with semantic (\sectref{sec:6-2}) and morphological (\sectref{sec:6-3}) properties of adjectives, we shall say something about their distribution and phonological structure. 



Normally adjectives do the job of modifying nouns, as in example (\ref{ex:6-1}), or that of descriptive predication, as in (\ref{ex:6-2}).

\begin{exe}
\ex
\label{ex:6-1}
\gll líi bíiḍ-i [ɡéeḍ-i] rusóx léed-i \\
he.\textsc{erg} very-\textsc{f} big-\textsc{f} power find.\textsc{pfv-f} \\
\glt `He gained very much power.' (B:ATI021)

\ex
\label{ex:6-2}
\gll pirsaahíb [ǰáand-u] de \\
Pir.Sahib alive-\textsc{msg} was  \\
\glt `Pir Sahib (a pious man) was (still) alive.' (B:ATI047)
\end{exe}

Occasionally, some adjectives may also occur on their own in noun phrases and then distributionally function as nouns (see \sectref{subsec:6-3-1}). On the other hand, another noun may also modify the head noun of a~noun phrase, in genitive as well as in its basic form, and also be used predicatively (see Chapter \sectref{chap:10}).


Structurally, inflected adjectives have a~lot in common with nouns and some of the most common finite verb forms (those historically based on participles). Compare, for instance, the words highlighted in examples (\ref{ex:6-3}) and (\ref{ex:6-4}) (the first an~adjective, the second a~noun, and the third a~verb), all with an~ending segment \textit{éeli} (which illustrates the partial alliterative agreement found in Palula, \citealt[87--88]{corbett2006}).

\begin{exe}
\ex
\label{ex:6-3}
\gll aní kuḍ [čéel-i] \\
\textsc{prox} wall[\textsc{f.sg}] thick-\textsc{f} \\
\glt `This wall is thick.' (B:DHE4885)

\ex
\label{ex:6-4}
\gll íṇc̣-a [čhéeli] [khéel-i] thaní \\
bear-\textsc{ob} goat[\textsc{f.sg}] eat.\textsc{pfv-f} \textsc{qt} \\
\glt `The bear has eaten the goat.' (A:PAS056)
\end{exe}

However, while gender is an~inherent property of nouns, gender marking of adjectives and verbs is only in the form of agreement with the gender of a~noun. 


The most typical Palula adjectives belong to the inflecting group and comprise a~monosyllabic stem and an~agreement suffix (see \sectref{subsec:6-3-1}). The stem typically consists of a~long first"=mora accented vowel, maximally preceded by a~two"=consonant onset and followed by a~two"=consonant cluster: \textit{tríimb-} `thick', minimally without an~onset but followed by a~stem consonant: \textit{áaḍ-} `half'. Almost equally common is a~monosyllabic stem consisting of a~short accented vowel, maximally preceded by a~two"=consonant onset and followed by a~two"=consonant cluster: \textit{tríṣṭ-} `bitter', minimally without an~onset but followed by a~stem consonant: \textit{úč-} `little'. There are also monosyllabic stems with a~long second"=mora accented vowel: \textit{mhoór-} `sweet', \textit{lhuúṇ-} `salty'. An interesting fact is that a~word structure resulting from this latter type of stem plus the obligatory agreement suffix is not uncommon for adjectives, i.e. a~second"=mora accented penultimate syllable~-- \textit{mhoóru}, whereas the same word structure is rather uncommon with nouns~-- \textit{mheéli} `mother'.


Adjectives belonging to the inflecting group, but with a~two"=syllable stem, are also frequent in the language. All such stems are accented on the vowel of the last syllable, which is either a~short vowel~-- as in \textit{bhakúl-} `fat'~-- or a~first"=mora accented long vowel~-- as in \textit{šidáal-} `cold'. The first unaccented vowel may be either long or short. I do not have any evidence of three"=syllable adjective stems in the inflecting group.


Non"=inflecting adjectives may be monosyllabic or polysyllabic. A number of the monosyllabic ones have a~CVC"=structure: \textit{šut} `sour', \textit{zeṛ} `yellow', \textit{paák} `clean', whereas the polysyllabic ones are structurally more diverse, though almost exclusively accented on the last syllable: \textit{muškíl} `hard, difficult', \textit{naawás} `dangerous', \textit{arzaán} `cheap', \textit{askóon} (B. \textit{askáan}) `easy'. 


\section{Semantic properties of adjectives}
\label{sec:6-2}

Properties or qualities that are typically coded or lexicalised as adjectives in many other languages generally occur as such in Palula as well. \citeauthor{dixon1982} concludes from a~comparative study, that adjectives with the semantic content of dimension, colour, age and value are the most likely to occur, however small the class of adjective is in a~language \citep[46]{dixon1982}. Along similar~-- but not identical~-- lines, \citet[82]{givon2001a} points out size, colour and a~number of qualities perceived by human senses as most typically expressed with adjectives. 



The inflecting adjectives exemplified in the following sections are given in their masculine singular form, and the invariant adjectives are consistently indicated with the abbreviation \textsc{inv}.


\subsection{Dimensional adjectives}
\label{subsec:6-2-1}

Many of the adjectives denoting spatial dimensions come as antonym pairs. It is not uncommon that the choice between near"=synonyms is dictated by animacy in one way or another:



\begin{table}[H]
\begin{tabularx}{\textwidth}{ l Q l Q }
\textit{ɡáaḍu} &
`big, grown, important' &
\textit{lhoóku} &
`small, young'\\
\textit{ɡháanu} &
`large' &
\textit{léku} &
`small'\\
\textit{čóolu} &
`thick, wide' &
\textit{čúṇu} &
`thin'\\
\textit{bhakúlu} &
`fat, thick' &
\textit{tróku} &
`thin, weak'\\
\textit{thúlu} &
`fat' &
\textit{khaṭáanu} &
`short'\\
\textit{dhríɡu} &
`long, tall' &
&
\\
\textit{čaáx} &
`fat, strong' (\textsc{inv)} &
&
\\
\end{tabularx}
\end{table}


As can be seen in the list, many of the adjectives here are of the inflecting type.


The antonym pair \textit{ɡáaḍu-lhoóku} has the additional connotation of age~-- further extended to relative importance when applied to human beings~-- which the pair \textit{ɡháanu-léku} lacks insofar as it simply refers to physical size. While the former word pair often occurs with humans, the latter pair tends to be used more frequently with animals and inanimates, but also for qualifying small children (as far as \textit{léku} is concerned).



The adjectives \textit{čóolu, bhakúlu, thúlu} and \textit{čaáx} all have to do with thickness, but here also additional semantic properties as well as the animacy of the noun being qualified governs their selection. Whereas \textit{čóolu} is used invariably to define the thickness of inanimate objects, \textit{čaáx} is never used to refer to anything other than animates, and is used primarily when talking about fat, strong and well"=fed domestic animals. The group of nouns being qualified by \textit{thúlu} is similar to that of \textit{čaáx}, but with less emphasis on the strength and more on the actual dimensions, whereas the ``in"=between'' adjective \textit{bhakúlu} is very widely applied to humans and animals, as well as inanimates. 



The opposite (and in this culture, negatively rather than positively associated) terms available in this case (\textit{čúṇu} and \textit{tróku}) are fewer, and as expected, each has a~wider scope than any of the aforementioned antonyms; \textit{tróku} is used at the upper end of the animacy continuum and \textit{čúṇu} at the lower end. The pair \textit{dhríɡu-khaṭáanu} refers to length as well as to vertical extension.


\subsection{Colour adjectives}
\label{subsec:6-2-2}

Adjectives denoting brightness (dark/light, black/white) come in antonym pairs, whereas the other colour terms cover particular sections of the rainbow and, as such, similar to other languages, form a~complement set with a~very restricted size \citep[19, 46]{dixon1982}: 



\begin{table}[H]
\begin{tabularx}{\textwidth}{ l@{\hspace{20pt}} Q l@{\hspace{20pt}} Q }
\textit{k(r)iṣíṇu} &
`black' &
\textit{paṇáaru} &
`white'\\
\textit{lhóilu} &
`red, (brown)' &
\textit{c̣hiṇ} &
`dark' (\textsc{inv)}\\
\textit{níilu} &
`blue, green' &
\textit{práal} &
`light' (\textsc{inv)}\\
\textit{zeṛ} &
`yellow' (\textsc{inv)} &
\textit{aaɡabháanu} &
`sky"=coloured'\\
\end{tabularx}
\end{table}


Many of these adjectives are of the inflecting type, but hardly all of them.



Including the terms for `black' and `white', there are four basic colour"=terms in Palula: \textit{k(r)iṣíṇu, paṇáaru, níilu, lhóilu.} For reasons we will return to below, \textit{zeṛ} and \textit{aaɡabháanu} must be considered less basic and less typical than other Palula adjectives, and among them \textit{zeṛ} at least must be considered a~recent Pashto loan (\textit{ziyaṛ} `yellow; brass', \citealt{raverty1982}). The term \textit{níilu} covers the whole spectrum referred to with English `blue' and `green', but the particular nuance can be specified with nominal derivations, such as \textit{aaɡabháanu}. The pair \textit{c̣hiṇ-práal} act as rather marginal adjectives; they tend to be used as nouns, `darkness' and `light', respectively.



The occurrence of the four basic terms, as well as the acquisition of new terms, confirms the hierarchy and distributional restrictions suggested by \citet[2--5]{berlinkay1969}. That yellow (and perhaps light colours in general) used to be associated with \textit{paṇáaru} `white' is supported by historical"=comparative data, where the cognate of this in a~number of NIA languages, in some MIA languages as well as in the OIA use of \textit{p\'{\={a}}ṇḍara-} (in Šatapatha"=Brahmāṇa), covers senses such as `whitish"=yellow, yellow, pale, white' \citep[8047]{turner1966}. That \textit{paṇáaru}, at least in the not too distant past, was associated with the colour of maize is hinted at in the present use of the derived verb \textit{paṇará} `rinsing or peeling the ears of corn, white"=wash'. Similarly for \textit{k(r)iṣíṇu} `black', historical"=comparative data suggests a~past usage covering dark colours in general, or more specifically `dark blue, black' as attested for OIA \textit{kr̥ṣṇá-} (in Rgveda, according to \citealt[3451]{turner1966}). 



Composite categories of blue/black and yellow/white, respectively, unattested in the previous findings of \citet{berlinkay1969}, have been attested in the more recent findings of \citet[17]{kayetal1991}. Following their reasoning, a~first stage in the evolution of basic colour terms would consist of two composites \citep[19]{kayetal1991}, one white/yellow/red and one green/blue/black, essentially embodying a~distinction between ``light'' colours and ``dark'' colours and, in turn, would correspond to a~universal or near"=universal distinction between day (light time) and night (dark time) \citep[288]{wierzbicka1996}. These two super"=categories may perhaps roughly correspond to a~presumed historical ``wider'' usage of the cognates of \textit{paṇáaru} and \textit{k(r)iṣíṇu}, respectively. 



\begin{table}[ht]
\caption{Hypothetical evolution of Palula colour terms}
\begin{tabularx}{\textwidth}{ Q Q l Q Q l Q Q }
\lsptoprule
Stage 1 &
&
&
Stage 2 &
&
&
Stage 3 &
\\\hline
\textit{k(r)iṣíṇu} &
(bk/bu/g) &
 &
\textit{k(r)iṣíṇu} &
(bk) &
&
&
\\
&
&
 &
\textit{níilu} &
(bu/g) &
&
&
\\
\textit{paṇáaru} &
(w/y/r) &
&
\textit{lhóilu} &
(r) &
&
&
\\
&
&
&
\textit{paṇáaru} &
(w/y) &
&
\textit{zeṛ} &
(y)\\
&
&
&
&
&
&
\textit{paṇáaru} &
(w)\\\lspbottomrule
\end{tabularx}
\label{tab:6-1}
\end{table}


That would also assume that the other two basic terms have entered into the system some time later
and only gradually gained their status as basic and~-- in relation to the first two~-- completely
complementary terms. The cognate of \textit{níilu} is known also in OIA, \textit{nī la-}
\citep[7563]{turner1966}, which refers to `dark blue, dark green, black', thus competing with OIA
\textit{kr̥ṣṇá-} but also used as a~noun `blue substance; indigo'. This probably means that it
started out as a~noun specifically referring to a~particular plant and then gradually gained its
more general use to signify the colour of that plant before becoming associated with the blue/green
segment of the blue/green/black composite. The etymology for \textit{lhóilu} seems somewhat shakier,
but \citet[11168, 11158]{turner1966} connects it with an~OIA root \textit{lōhá-}, variously glossed
as `red, copper"=coloured; made of copper; copper; iron'. This, at least, hints at a~background as
a~noun even for this term, suggesting that it has gradually gained status as a~basic colour term,
possibly competing with a~former white/yellow/red composite.



Somewhat speculatively we may therefore make some guesses as to the evolution of colour terms in previous stages of Palula, as displayed in \tabref{tab:6-1}.



That the blue/green composite is a~persistent one is confirmed by some languages in which this composite remains undissolved even after brown and purple have become thoroughly introduced into the system \citep[18]{kayetal1991}.



It is obvious from this discussion that it is the less basic and relatively newly acquired colour adjectives that are invariant, such as \textit{zeṛ} `yellow', whereas the most basic and firmly established terms, such as \textit{kriṣíṇu} `black' and \textit{paṇáaru} `white', are inflecting. 


\subsection{Age adjectives}
\label{subsec:6-2-3}


The size of the age type of adjectives is, like in many other languages, very restricted \citep[46]{dixon1982}. Those that occur come in antonym pairs, similar to the dimensional adjectives:



\begin{table}[H]
\begin{tabularx}{\textwidth}{ l@{\hspace{20pt}} Q l@{\hspace{20pt}} Q }
\textit{puróoṇu} &
`old' (inanimate) &
\textit{náawu} &
`new'\\
\textit{búuḍu} &
`old' (animate) &
\textit{zuwaán} &
`young' (\textsc{inv)}\\
\textit{ɡhaḍeeró (ɡaḍheeró)} &
`older' &
\textit{lhookeeró} &
`younger'\\
\end{tabularx}
\end{table}


The antonym pair \textit{búuḍu}~--\textit{zuwaán} are perhaps used more as independent nouns in the sense of `old man' and `young man', respectively, especially the latter, than as adjectives modifying another noun. 



The pair \textit{ɡhaḍeeró-lhookeeró} reveals a~trace of an~earlier comparative formation, those forms being old comparatives of \textit{ɡáaḍu} `big, grown, important', and \textit{lhoóku} `small, young' (see above). In this case, however, it is solely the age component that remains, as in \textit{ɡhaḍeeró bhroó} `elder brother'. In particular, \textit{ɡhaḍeeró} is used as a~noun, and often in a~more restricted sense, meaning an `elder' of an~entire village, clan or tribe.


\subsection{Value adjectives}
\label{subsec:6-2-4}


While \citet[46]{dixon1982} places a~semantic type ``value'' among those most likely to belong to the adjectives, even when the class is small, \citet[83]{givon2001a} includes what he calls ``evaluative'' in the less prototypical group:



\begin{table}[H]
\begin{tabularx}{\textwidth}{ l@{\hspace{30pt}} Q l@{\hspace{30pt}} l }
\textit{šóo} &
`good' &
\textit{kháaču} &
`bad'\\
\textit{šišówu} &
`beautiful' &
\textit{beetseerá} &
`ugly' (\textsc{inv)}\\
\textit{sóoru} &
`healthy, whole' &
\textit{naawás} &
`difficult, dangerous' (\textsc{inv)}\\
\textit{askóon} &
`easy' (\textsc{inv)} &
\textit{zoór} &
`difficult' (\textsc{inv)}\\
\textit{paák} &
`clean' (\textsc{inv)} &
\textit{muškíl} &
`difficult' (\textsc{inv)}\\
\textit{arzaán} &
`cheap' (\textsc{inv)} &
\textit{ɡ(i)raán} &
`expensive' (\textsc{inv)}\\
\end{tabularx}
\end{table}


In Palula, one of the more obvious characteristics of this type is that to a~large extent it is made up of relatively recent loans, and many of them are morphologically invariant.


\subsection{Physical"=property adjectives}
\label{subsec:6-2-5}


Lacking a~better term, I chose to use the term suggested by \citet[16]{dixon1982} for properties
that in different ways parallel human sensory capacities. What is essentially the same class of
potential adjectives, \citet[82]{givon2001a} divides into four groups: auditory qualities, shape,
taste and tactile. For reasons having to do with their lower temporal stability,
\citet[83]{givon2001a} places some of \citeauthor{dixon1982}'s physical properties under transitory
states (e.g. temperature and external condition), a~group that \citeauthor{givon2001a} considers
less prototypically coded as adjectives:



\begin{table}[H]
\begin{tabularx}{\textwidth}{ l@{\hspace{30pt}} Q l@{\hspace{30pt}} Q }
\textit{piṇḍúuru} &
`round' &
\textit{lhuúṇu} &
`salty'\\
\textit{mhoóru} &
`sweet' &
\textit{tríṣṭu} &
`bitter'\\
\textit{kooṭáatu} &
`hard' &
\textit{koomáalu} &
`soft'\\
\textit{tíiṇu} &
`sharp' &
\textit{búku} &
`dull'\\
\textit{táatu} &
`warm' &
\textit{šidáalu} &
`cold'\\
\textit{páaku} &
`ripe, cooked' &
\textit{óomu} &
`unripe, raw'\\
\textit{šúku} &
`dry' &
\textit{síndu} &
`wet'\\
\end{tabularx}
\end{table}

As can be seen, and also pointed out by \citet[19]{dixon1982}, the properties of this type that code taste seem to be members of a~complement set akin to that of colour terms (which in themselves very well could be treated as physical properties that are visually perceived), whereas some of the other clusters (like the tactile suggested by \citealt[82]{givon2001a}) occur in antonym pairs in a~fashion similar to that of dimensional adjectives. 


While all the aforementioned types (dimensional, colour, age and value) are among those typically occurring as adjectives, according to \citet[46]{dixon1982}, this type occur with less frequency in languages with smaller inventories of adjectives.



Whereas tactile- and taste"=related properties are rather well"=represented, auditory qualities and
more specialised shape (i.e. visually perceived) properties are generally not coded as adjectives in
Palula. Some of those signifying less temporally stable properties, such as \textit{šúku} `dry' and
\textit{páaku} `ripe, cooked' are less adjectival; both are in fact formally identical with
Perfective (and Perfective Participle) forms of the verbs \textit{šuš-} `dry' and \textit{pač-}
`ripen, become cooked', respectively.



However, it is interesting to note that all of the exemplified adjectives of this type are inflecting.


\subsection{Speed adjectives}
\label{subsec:6-2-6}


It is uncertain whether this type really qualifies as a~main adjective type, or rather ought to be included among physical property adjectives, but in order to compare with \citeauthor{dixon1982}'s results I have kept it separate. As predicted by \citet[46]{dixon1982}, it is a~very restricted type with respect to size:



\begin{table}[H]
\begin{tabularx}{\textwidth}{ l@{\hspace{20pt}} Q l@{\hspace{20pt}} Q }
\textit{teéz} &
`fast, strong' (\textsc{inv)} &
\textit{bhraáš} &
`slow' (\textsc{inv)}\\
\end{tabularx}
\end{table}


It is doubtful how much these words are really used as adjectives. Mostly they are used as adverbs, qualifying verbs rather than nouns. Used as an~adverb, \textit{teéz}, for instance, has at least two near"=synonyms, \textit{lap} (or \textit{lab}) and \textit{táru}, both meaning `quickly, fast'. \citet[47--48]{dixon1982} notes a~certain conditional relationship between the physical property and the speed type: if the physical property type in a~given language is associated with the verb type (which we saw above that it to some extent is), then the speed type will be associated with the class of adverbs.


\subsection{Human"=propensity adjectives}
\label{subsec:6-2-7}


A type of adjective that tends to be particularly numerous is one that is applied to human behaviour and states. Although primarily qualifying nouns referring to humans, it is often extended to include all higher animates \citep[16, 46]{dixon1982}. The distinction between this type and value adjectives (see above) is at best approximate. Many of the properties found here correspond to \citeauthor{givon2001a}'s (\citeyear[83]{givon2001a}) transitory states and states of living, both of which less prototypically occur as adjectives. Some of them have more or less obvious antonyms, but for others, there is not one clear opposite term, or it would simply not make sense to talk about an~opposite trait:



\begin{table}[H]
\begin{tabularx}{\textwidth}{ l@{\hspace{30pt}} Q l@{\hspace{30pt}} Q }
\textit{sóoru} &
`whole, healthy' &
\textit{bidráaɡu} &
`sick'\\
\textit{ac̣híiwu} &
`sane' &
\textit{kaantíiru} &
`mad'\\
\textit{bhúuru} &
`deaf' &
\textit{ṣíiṛu} &
`blind'\\
\textit{čaaṛaá} &
`dumb' &
\textit{khúšu} &
`left"=handed'\\
\textit{lheéṇḍu} &
`bald' &
\textit{khúṭu} &
`lame'\\
\textit{bhiiroó} &
`male' &
\textit{súutri} &
`female'\\
\textit{khíndu} &
`tired' &
&
\\
\end{tabularx}
\end{table}


Whereas languages with a~very small class of adjectives tend to associate physical property with verbs, human propensity instead tends to be associated with nouns. In fact, a~number of the Palula adjectives of this type that have to do with (more permanent) states, can equally well function as nouns and often do; e.g. \textit{ṣíiṛu} could have the sense `blind' when modifying a~noun and the sense `blind person' as the head of a~noun phrase. On the other hand, some other adjectives presented here that are used for (more temporary) mental"=internal states are closely associated~-- or even identical~-- with verb forms, e.g. \textit{khíndu}, which is the Perfective (or Perfective Participle) of \textit{khínǰ-} `become tired'.


\subsection{Adjectival quantifiers}
\label{subsec:6-2-8}

Another group of Palula adjectives with no counterpart either in \citeauthor{givon2001a}'s or \citeauthor{dixon1982}'s discussion is made up of quantifying or somewhat existential expressions. Some of them are distributionally rather flexible:



\begin{table}[H]
\begin{tabularx}{\textwidth}{ l@{\hspace{20pt}} Q l@{\hspace{20pt}} l }
\textit{bíiḍu} &
`many, much' &
\textit{úču} &
`few, a~little'\\
\textit{biǰóolu} &
`several' &
\textit{phalúuṛu} &
`sole, only'\\
\textit{áaḍu} &
`half' &
\textit{púuntu} &
`full'\\
\textit{xaalí} &
`empty (also: pure, whole)' (\textsc{inv)} &
\textit{puunǰí} &
`full' (\textsc{inv})\\
\textit{falaankí} &
`unknown, so and so' (\textsc{inv)} &
&
\\
\end{tabularx}
\end{table}


Both \textit{púuntu} `full' and \textit{puunǰí} `full' are essentially verbal, the first being the Perfective Participle of \textit{púunǰ-} `become filled' and the second the Converb of the same verb.


\subsection{Summary of findings}
\label{subsec:6-2-9}

The investigation in Sections \sectref{subsec:6-2-1}~--\sectref{subsec:6-2-8} gives us a~somewhat clearer idea how to view the class we have referred to as adjectives in Palula. 



First, there is a~core of typical adjectives, used exclusively or nearly exclusively as descriptive or qualitative modifiers of nouns. Many of those are found among words referring to dimension and colour. 



Second, there are a~number of overlaps, and some words from the following categories can also be used as modifiers of nouns: a) adverbs (i.e. as modifiers or adjuncts of verb phrases or adjective phrases), particularly words referring to speed or quantity; b) nouns, particularly words referring to age and human propensity, but also some of those referring to dimension, quantity, age and colour; c) verbs (or more specifically participles), particularly words referring to physical property but also some of those referring to quantity and human propensity.



Third, while the firmly established adjectives (used with relative freedom attributively as well as predicatively) tend to be of the inflecting type, most newly acquired words used as noun modifiers are invariant and usually more restricted in their distribution (more readily used predicatively than attributively), especially newly acquired evaluative words. The relative age of the colour terms discussed in \sectref{subsec:6-2-2}, also illustrates this difference: while the older and most basic terms are inflecting adjectives, the relatively newly acquired colour terms are less basic, distributionally more restricted and invariant.



The relative closeness between important groups of adjectives and nouns on the one hand and verbs on the other suggests that the diachronic rise of agreement features within the noun phrase (which are to some extent alliterative, \citealt[87]{corbett2006}) is to be sought either in the nominal paradigm or in the agreement features of participles, or perhaps both, thus reinforcing one another and possibly strengthened further through analogy.


\section{Morphological properties of adjectives}
\label{sec:6-3}

\subsection{Inflectional morphology}
\label{subsec:6-3-1}

As has been mentioned already, adjectives can be divided into two main categories as far as morphology is concerned: inflecting and invariant (or non"=inflecting) adjectives. When they qualify nouns, as modifiers or as predicative complements, inflecting adjectives show agreement in gender, number and case. Although the inflectional morphology in many ways mirrors that of nouns, adjectival inflection shows a~considerably lower degree of declensional variation and fewer available forms. Case agreement is, for instance, part of the inflectional properties of adjectives, but it plays a~minor part in the paradigm as compared to the nominal paradigm, and number contrast is only partially displayed.

\subsubsection*{Inflecting adjectives}

The great majority of inflecting adjectives occur in three forms ending in \textit{-u, -a} and \textit{-i}, respectively, as displayed in \tabref{tab:6-2}. (There is also a~marginal \textsc{fpl} form ending in \textit{-im}, although mainly realised when an~adjective is being substantivised, \sectref{subsec:6-3-2}.)


\begin{table}[ht]
\caption{Inflection of adjectives}
\begin{tabularx}{\textwidth}{ Q Q l l }
\lsptoprule
Nominative masculine singular &
Nominative masculine plural/oblique masculine &
Feminine &
\\\hline
\textit{dhríɡ-u} &
\textit{dhríɡ-a} &
\textit{dhríɡ-i} &
`long, tall'\\\lspbottomrule
\end{tabularx}
\label{tab:6-2}
\end{table}


The first two forms are a~direct reflection of the nominative singular and the nominative plural/oblique singular forms of the u-marked nouns of the \textit{a}"=declension, and the feminine forms correspond to the singular and plural forms of the \textit{m}-declension nouns.


Inflecting adjectives with an~accent on a~long \textit{óo} ({\textless} áa) or
\textit{áa} ({\textless a}) in \textsc{nom.msg} undergo vowel modification (umlaut) to \textit{ée} in the feminine agreement forms with an~i-suffix (\tabref{tab:6-3}).


\begin{table}[ht]
\caption{Inflection (involving umlaut) of adjectives}
\begin{tabularx}{\textwidth}{ l@{\hspace{25pt}} l@{\hspace{25pt}} l@{\hspace{25pt}} Q }
\lsptoprule
\textsc{Nom.msg} &
\textsc{Nom.mpl}/\textsc{Ob.m} &
Feminine &
\\\hline
\textit{ɡáaḍu} &
\textit{ɡáaḍa} &
\textit{ɡéeḍi} &
`big, important, grown'\\
\textit{paṇáaru} &
\textit{paṇáara} &
\textit{paṇéeri} &
`white'\\
\textit{áaḍu} &
\textit{áaḍa} &
\textit{éeḍi} &
`half'\\
\textit{óomu} &
\textit{óoma} &
\textit{éemi} &
`raw, unripe'\\
\textit{puróoṇu} &
\textit{puróoṇa} &
\textit{puréeṇi} &
`old'\\
\textit{sóoru} &
\textit{sóora} &
\textit{séeri} &
`healthy, whole'\\\lspbottomrule
\end{tabularx}
\label{tab:6-3}
\end{table}


There are individual adjectives inflecting differently (\tabref{tab:6-4}), but they are too few to analyse meaningfully as parts of separate declensions; however, it should be pointed out that they do bear resemblances to some of the other noun declensions and may possibly reflect earlier more widespread declensional differences among adjectives, similar to the adjective classes in Kohistani Shina \citep[100--103]{schmidtkohistani2008}.


\begin{table}[ht]
\caption{Irregularly inflecting adjectives}
\begin{tabularx}{\textwidth}{ Q Q Q Q }
\lsptoprule
\textsc{Nom.msg} &
\textsc{Nom.mpl}/\textsc{Ob.m} &
Feminine &
\\\hline
\textit{šóo} &
\textit{šóo-a} &
\textit{šu(y)-i} &
`good'\footnote{In the Biori dialect this adjective has been reinterpreted as an~invariant adjective: \textit{šúi}.}\\
\textit{bhiiroó} &
\textit{bhiireé} &
-- &
`male'\\
\textit{čaaṛaá} &
\textit{čaaṛaá} &
\textit{čaaṛái} &
`dumb'\\\lspbottomrule
\end{tabularx}
\label{tab:6-4}
\end{table}

\subsubsection*{Invariant adjectives}

Adjectives that do not inflect for agreement with a~noun head either end in a~consonant or a~vowel other than \textit{u}.


The following are examples of invariant (non"=inflecting) adjectives:



\begin{table}[H]
\begin{tabularx}{\textwidth}{ l@{\hspace{30pt}} Q l@{\hspace{30pt}} Q }
\textit{ḍanɡ} &
`hard' &
\textit{takṛá} &
`strong'\\
\textit{zoór} &
`difficult' &
\textit{ḍhíla} &
`loose'\\
\textit{šum} &
`stingy' &
\textit{ṭeéṭ} &
`tight'\\
\textit{xaróob} &
`bad' &
\textit{ɣoṛ} &
`greasy'\\
\textit{saká} &
`real' &
\textit{taaqatwár} &
`powerful'\\
\end{tabularx}
\end{table}


Although most of these without doubt function as adjectives, there are a~number of features that make at least a~fair number of them slightly less prototypical: A large proportion of the invariant adjectives are fairly recent loans from other languages, and although some of them are old enough in the language to have acquired an~indigenised phonology (such as \textit{xaróob} from the Perso"=Arabic \textit{xarāb}), they have not yet developed the inflectional paradigm typical of most inflecting adjectives. Furthermore, many of these tend to be used predicatively, whereas the inflecting adjectives are used equally well attributively and predicatively. They can also to a~larger extent be used as nouns and adverbs apart from their adjectival usage.


\subsection{Substantivisation}
\label{subsec:6-3-2}

An adjective can occur on its own as the head of a~noun phrase when substantivised, i.e. an~adjective like \textit{ɡáaḍu} `grown, big, important' thus being used as in (\ref{ex:6-5}) in the sense `the big one, the adult'. Apart from a~slight semantic shift, being substantivised also means that case forms otherwise only applied to nouns have to be used (when applicable). This is also the realm where the otherwise rare feminine plural forms with \textit{-im}, as in (\ref{ex:6-6}), are non"=optional.

\begin{exe}
\ex
\label{ex:6-5}
\gll so [ɡáaḍ-am] díi náqal th-áan-u \\
\textsc{3msg.nom} big-\textsc{mpl.ob} from copying do-\textsc{prs"=msg} \\
\glt `He is imitating the adults (the grown up men).' (A:SMO005)

\ex
\label{ex:6-6}
\gll se [éeḍ-im] bhíilam khonḍíl-im \\
\textsc{def} half-\textsc{fpl} fearfully speak.\textsc{pfv"=fpl} \\
\glt `The rest of them spoke fearfully.' (A:BEZ022)
\end{exe}

\subsection{Comparison of degree}
\label{subsec:6-3-3}

As we saw above, there are some traces of an~earlier inflection for comparison, and \citet[17]{morgenstierne1941} also points out some remains of the OIA superlative degree, but in the modern language these degrees are exclusively expressed periphrastically. The comparative is expressed by a~standard of comparison in oblique case and the postposition \textit{díi} `from', as illustrated in (\ref{ex:6-7}) and (\ref{ex:6-8}), preceding the adjective functioning as the parameter of comparison, literally translatable as `X is large from Y'. 

\begin{exe}
\ex
\label{ex:6-7}
\gll bhiúuṛi [dhamareet-á díi ɡáaḍ-u] déeš \\
Biori Dhamaret-\textsc{ob} from large-\textsc{msg} village \\
\glt `Biori is a~larger village than Dhamaret.' (B:DHE4803)

\ex
\label{ex:6-8}
\gll [kúuk-a díi] kúuk-e putr [hušiaár] \\
crow-\textsc{ob} from crow-\textsc{gn} son wise  \\
\glt `The son of a~crow is wiser than the crow [himself].' (B:PRB004)
\end{exe}

A construction having a~function close to the superlative of many European languages similarly uses the oblique form of the indefinite pronoun \textit{buṭheé} `all' and \textit{díi}, example (\ref{ex:6-9}), literally meaning something like `X is more powerful than all'.

\begin{exe}
\ex
\label{ex:6-9}
\gll [buṭhimeém díi taaqatwár] hín-u insaán \\
all-\textsc{ob} from powerful be.\textsc{prs"=msg} human \\
\glt `Man is the most powerful (creature).' (A:KIN003)
\end{exe}

\subsection{Derivational morphology}
\label{subsec:6-3-4}

Apart from the general flexibility of some words, as we saw above (\sectref{subsec:6-2-9}), being used as nouns \textit{and} adjectives or as nouns \textit{and} participials etc., adjectives may also be derived morphologically from other parts of speech, particularly from nouns and adverbs. A commonly occurring adjectival derivative suffix applied to nouns, particularly referring to materials or substances, is the accent"=bearing derivational suffix \textit{-íil}, as shown in \tabref{tab:6-5}, to which agreement suffixes are subsequently added.


\begin{table}[ht]
\caption{Adjectives derived from nouns}
\begin{tabularx}{\textwidth}{ Q l l P{22mm} l }
\lsptoprule
Derived adjective &
&
&
Noun derived\newline from &
\\\hline
\textit{šaak-íilu} &
`wooden' &
$\leftarrow$ &
\textit{šaák} &
`wood'\\
\textit{koow-íilu} &
`made of olive wood' &
$\leftarrow$ &
\textit{koó} &
`olive tree wood'\\
\textit{wíi-lu} &
`watery' &
$\leftarrow$ &
\textit{wíi} &
`water'\\
\textit{čimar-íilu, čeemáar-u} &
`of steel, iron' &
$\leftarrow$ &
\textit{čímar} &
`steel'\\
\textit{ǰhaṭ-íilu} &
`made of fur' &
$\leftarrow$ &
\textit{ǰhaáṭ} &
`fur'\\\lspbottomrule
\end{tabularx}
\label{tab:6-5}
\end{table}


The difference between the noun"=to"=adjective derivation in \textit{-íil} and the Perfective Participles with an \textit{íl}-segment should be noted, as they may take on slightly different semantics when used attributively: \textit{ǰhaṭíilu} `made of fur' vs. \textit{ǰhaṭílu} `fur"=clad, hairy'. A special group of adjectives is morphologically derived from the relatively small class of non"=derived (primarily calendrical) temporal adverbs (see \sectref{subsec:7-1-3}) by means of an~accent"=bearing suffix \textit{-úk} (\tabref{tab:6-6}), to which agreement suffixes are added.


\begin{table}[ht]
\caption{Adjectives derived from temporal adverbs}

\begin{tabularx}{\textwidth}{ Q l l Q l }
\lsptoprule
Derived adjective &
&
&
Adverb derived from &
\\\hline
\textit{aaǰ-úku} &
today's &
$\leftarrow$ &
\textit{áaǰ} &
today\\
\textit{dhooṛ-úku} &
yesterday's &
$\leftarrow$ &
\textit{dhoóṛ} &
yesterday\\
\textit{tip-úku} &
the present &
$\leftarrow$ &
\textit{típa} &
now\\
\textit{bhiaal-úku} &
last night's &
$\leftarrow$ &
\textit{bhióol} &
last night\\\lspbottomrule
\end{tabularx}
\label{tab:6-6}
\end{table}


\section{Numerals}
\label{sec:6-4}

The Palula numeral system is basically vigesimal, a~system common to most languages of the region \citep[823]{bashir2003}. The numerals discussed in this section are cardinal numerals, substantivised numerals, and ordinal numerals.


\subsection{Cardinal numerals}
\label{subsec:6-4-1}

The cardinals from 20 to 100 are vigesimal, which means that 20 (and not ten) functions as a~base, preceded by a~multiplier. The word \textit{bhiš} `20' is essentially a~noun in this construction, plural inflected and preceded by a~modifying lower numeral: 



\begin{table}[H]
\begin{tabularx}{\textwidth}{ Q Q l@{\hspace{50pt}} }

\textit{bhiš (purá bhiš)} &
`twenty' (`full twenty') &
\\
\textit{dúu bhiš-á (dubhišá)} &
`forty' &
2 x 20\\
\textit{tróo bhiš-á} &
`sixty' &
3 x 20\\
\textit{čúur bhiš-á} &
`eighty' &
4 x 20\\
\textit{páanǰ bhiš-á} &
`one hundred' &
5 x 20\\
\end{tabularx}
\end{table}


The numbers between 20, 40, and 60 etc. are formed by coordination, whereby a~coordinating element (cf. \sectref{subsec:13-2-1}) is attached to 20 and then followed by one of the numerals 1--19, as exemplified in (\ref{ex:6-10}) and (\ref{ex:6-11}). Phonologically, the complex is one word with one main accent. Note that the order is the reverse in some other languages in the region (cf. Kalam Kohistani, \citealt[57]{baart1999a}).

\begin{exe}
\ex
\label{ex:6-10}
\gll bhiš-ee-ṣó \\
twenty"=and"=six \\
\glt `26'

\ex
\label{ex:6-11}
\gll dubhiš-ee-ṣoṛíiš \\
two.twenty"=and"=sixteen \\
\glt `56'
\end{exe}

\tabref{tab:6-7} presents the Palula numerals from 1--40, the system after that being regular up to 100.


\begin{table}[ht]
\caption{Cardinal numerals}



\begin{tabularx}{.75\textwidth}{ l Q l Q l Q l Q }
\lsptoprule
1--20 &
&
21--40 &
&
\\\hline
1 &
\textit{áak, áa} &
21 &
\textit{bhiš-ee-áak} \\
2 &
\textit{dúu} &
22 &
\textit{bhiš-ee-dúu} \\
3 &
\textit{tróo} &
23 &
\textit{bhiš-ee"=tróo} \\
4 &
\textit{čúur} &
24 &
\textit{bhiš-ee-čúur} \\
5 &
\textit{páanǰ} &
25 &
\textit{bhiš-ee-páanǰ} \\
6 &
\textit{ṣo} &
26 &
\textit{bhiš-ee-ṣó} \\
7 &
\textit{sáat} &
27 &
\textit{bhiš-ee-sáat} \\
8 &
\textit{áaṣṭ} &
28 &
\textit{bhiš-ee-áaṣṭ} \\
9 &
\textit{núu} &
29 &
\textit{bhiš-ee-núu} \\
10 &
\textit{dáaš} &
30 &
\textit{bhiš-ee-dáaš} \\
11 &
\textit{akóoš} &
31 &
\textit{bhiš-ee"=akóoš}\\
12 &
\textit{bóoš} &
32 &
\textit{bhiš-ee-bóoš}\\
13 &
\textit{tríiš} &
33 &
\textit{bhiš-ee"=tríiš}\\
14 &
\textit{čandíiš} &
34 &
\textit{bhiš-ee-čandíiš}\\
15 &
\textit{panǰíiš} &
35 &
\textit{bhiš-ee"=panǰíiš}\\
16 &
\textit{ṣoṛíiš} &
36 &
\textit{bhiš-ee-ṣoṛíiš}\\
17 &
\textit{satóoš} &
37 &
\textit{bhiš-ee"=satóoš}\\
18 &
\textit{aṣṭóoš} &
38 &
\textit{bhiš-ee-aṣṭóoš}\\
19 &
\textit{aṇabhíš} &
39 &
\textit{bhiš-ee-aṇabhíš}\\
20 &
\textit{bhíš} &
40 &
\textit{dubhišá}\\\lspbottomrule
\end{tabularx}
\label{tab:6-7}
\end{table}


Although even the numerals one hundred and above can be given according to the vigesimal system, this is no longer in common use. Instead, the numerals 100, 200, etc., 1000, 2000, etc., and all higher numbers are represented by loans from languages of wider communication, essentially words from Pashto:



\begin{table}[H]
\begin{tabularx}{\textwidth}{ l@{\hspace{40pt}} Q }
100 &
\sffamily \textrm{\textit{áak sóo (paanǰ-bhišá)}} \\
200 &
\sffamily \textrm{\textit{dúu sóo/sówa (dáaš bhišá)}}\\
1 000 &
\sffamily \textrm{\textit{áak zir}}\\
2 000 &
\sffamily \textrm{\textit{dúu zára}}\\
10 000 &
\sffamily \textrm{\textit{dáaš zára}}\\
100 000 &
\sffamily \textrm{\textit{áa láak}}\\
\end{tabularx}
\end{table}

With a~growing emphasis on education in the Palula"=speaking area, even small children are familiar
with Urdu numerals and use them in free variation with the native Palula words, especially for all
numbers above ten. For indicating years according to the common era, Urdu numerals are used
exclusively.

Reduplication, as in (\ref{ex:6-12}), is used to form distributive numerals.

\begin{exe}
\ex
\label{ex:6-12}
\gll tus [aak-áak] looṛíi-a aṭ-óoi \\
\textsc{2pl.nom} \textsc{red}-one bowl-\textsc{pl} bring-\textsc{imp.pl} \\
\glt `Go and get a~bowl each [all of you]!' (A:KAT125-6)
\end{exe}

Emphasis or exclusiveness can be added to a~numeral with a~suffix \textit{-ee} (cf. emphasised pronouns, \sectref{sec:5-1}), as in (\ref{ex:6-13}), giving the meaning `only ten, etc.'

\begin{exe}
\ex
\label{ex:6-13}
\gll eesé waxt-íi [dáaš-ee] kušúni de \\
\textsc{rem} time-\textsc{gn} ten-\textsc{excl} household be.\textsc{pst} \\
\glt `In those days there were only ten households.' (A:PAS010)
\end{exe}

For inclusiveness, on the other hand, i.e. to give the meaning `all ten, etc.', a~form of the numeral with an~accented ending \textit{eé} and the numeral stem deaccented (cf. the ordinal numerals, \sectref{subsec:6-4-3}) is used, as can be seen in (\ref{ex:6-14}).

\begin{exe}
\ex
\label{ex:6-14}
\gll šíin"=ii čureé šeenbóo-a phooṭíl-a \\
bed-\textsc{gn} four.\textsc{incl} leg.of.bed-\textsc{pl} break.\textsc{pfv"=mpl} \\
\glt `All four legs of the bed broke.' (A:GHU024)
\end{exe}

\subsection{Substantivised numerals}
\label{subsec:6-4-2}


Apart from being used attributively, cardinal numbers, at least the lower ones, can also be used independently, as heads of noun phrases (like many adjectives and demonstratives). Used this way they are inflected for case, (\ref{ex:6-15})--(\ref{ex:6-16}).

\begin{exe}
\ex
\label{ex:6-15}
\gll [áak"=ii] ta ḍheerdáṛ nikhéet-i [áak"=ii] ba  aní phyaaṛmaǰ-í wée
breéx nikhéet-i \\
one-\textsc{gn} \textsc{prt} stomach.pain come.out.\textsc{pfv-f} one-\textsc{gn} \textsc{prt} \textsc{prox} side-\textsc{ob} in pain come.out.\textsc{pfv-f} \\
\glt `One of them started to feel pain in his stomach, whereas the other felt it in his side.'
(A:GHA058-9)

\ex
\label{ex:6-16}
\gll eesé [dašúm] maǰi dúu bhraawú de  \\
\textsc{rem} ten.\textsc{ob} among two brother.\textsc{pl} be.\textsc{pst}  \\
\glt `Among these ten were two brothers.' (A:PAS011)
\end{exe}

\subsection{Ordinal numerals}
\label{subsec:6-4-3}

Ordinal numbers (\tabref{tab:6-8}) are formed with an~accented and somewhat variable suffix \textit{-úma/-íma/-íima}, added to a~de"=accented form of the cardinal numerals. The ordinal `first' is altogether suppletive (but a~form \textit{akúme} has been noted in the B. dialect).


\begin{table}[ht]
\caption{Ordinal numerals}

\begin{tabularx}{\textwidth}{ l@{\hspace{20pt}} Q l@{\hspace{20pt}} Q }
\lsptoprule
1--10 &
&
11--20 &
\\\hline
\textit{aaweelíi} &
first &
\textit{akaašúma} &
11th\\
\textit{dhuíima, dhuyáama} &
second &
\textit{baašúma} &
12th\\
\textit{trayíma, trayáama} &
third &
\textit{treešúma} &
13th\\
\textit{čuríma} &
fourth &
\textit{čandeešúma} &
14th\\
\textit{panǰúma} &
fifth &
\textit{panǰeešúma} &
15th\\
\textit{ṣuyíma} &
sixth &
\textit{ṣuṛeešúma} &
16th\\
\textit{satúma} &
seventh &
\textit{sataašúma} &
17th\\
\textit{aṣṭúma} &
eighth &
\textit{aṣṭaašúma} &
18th\\
\textit{nuyíma} &
ninth &
\textit{aṇabhišúma} &
19th\\
\textit{dašúma} &
tenth &
\textit{bhišúma} &
20th\\\lspbottomrule
\end{tabularx}
\label{tab:6-8}
\end{table}

