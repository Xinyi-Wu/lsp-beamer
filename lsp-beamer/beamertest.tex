\documentclass[handout]{beamer}

\usepackage[utf8x]{inputenc}
\usepackage{textcomp}

\useoutertheme{lsp}

\usepackage{lsptitle}

\def\two@digits#1{\ifnum#1<10 0\fi\number#1}
\def\mytoday{\two@digits{\number\day}.\two@digits{\number\month}.\number\year}


\usepackage{xspace,multicol}
\newcommand{\latex}{\LaTeX\xspace}


\newcounter{lastpagemainpart}
\footnotesep0pt
\renewcommand{\footnoterule}{}
\usefootnotetemplate{
  \noindent
  \insertfootnotemark\insertfootnotetext}

\let\beamerfn=\footnote
\renewcommand{\footnote}[1]{%
\let\oldfnsize=\footnotesize%
\let\footnotesize=\tiny%
\beamerfn<\thebeamerpauses->{#1}%
\let\footnotesize=\oldfnsize}


\date{05.03.2014}

\usepackage{eurosym} 
\usepackage{ogonek}  % Dabrowska
% \usepackage{libertine}

% Irgendein Font definiert mir das \dh wieder über.
\renewcommand{\dh}{d.\,h.\xspace}        % was macht dh sonst??
\renewcommand{\centerline}[1]{\hfill#1\hfill\hfill\mbox{}}


\title{Language Science Press}
\institute{FU Berlin}
\author{Stefan Müller}



\begin{document}
\lspbeamertitle
\section{Hintergrund}

\frame{
\frametitle{Hintergrund}

\begin{itemize}[<+->]
\item 06.2012 Treffen mit Adele Goldberg, Thomas Herbst, Anatol Stefanowitsch
\item 08.2012 OALI-Webseite, Mails an prominente Kollegen
\item 08.2012 Treffen mit Martin Haspelmath
\item 10.2012 Kick-Off-Treffen an der FU-Berlin mit internationaler Beteiligung via Skype (Anke
  Lüdeling, Gereon Müller, FU-Linguistik, \ldots)
\item 02.2013 Einreichung Antrag im DFG-Programm \emph{Wissenschaftliche Monographien und monographische Serien im Open Access}
\item 12.2013 DFG-Finanzierungszusage, 2 von 17 Projekten gefördert, $>$~575,000 € für zwei Jahre,
  72\,\% der Gesamtfördersumme
\item 02.2014 Einstellung Sebastian Nordhoff
\item 03.2014 Publikation der ersten Bücher
\end{itemize}

}

\section{Organisation}

\frame{
\frametitle{Gemeinschaftsprojekt}

\begin{itemize}[<+->]
\item 426 Unterstützer, über 100 sehr prominent, viele anwesend

siehe: \url{http://langsci-press.org/Meta/sign/supporters}

\item Unterschreibe auch du!

\url{http://langsci-press.org/Meta/sign/}

\item dezentrale Organisation

\begin{itemize}
\item nur Bücher in Reihen
\item Reihenherausgeber übernehmen Verantwortung und steuern Ressourcen bei
\item Reihenvorschläge werden vom Advisory Board evaluiert
\item Satz, Korrekturlesen durch Herausgeber bzw.\ die Community
\end{itemize}

\item Speicherung, Archivierung und Katalogisierung durch FU-Bibliothek

\item Print on Demand mit verschiedenen Anbietern und Vertrieb über bekannte Kanäle
\end{itemize}


}


\frame[shrink=20]{
\frametitle{Advisory Board}% (alle Regionen und gender-balanced)}

\begin{multicols}{2}
\begin{itemize}
\item    Artemis Alexiadou (Stuttgart)
\item    Jim Blevins (U Cambridge)
\item    Balthasar Bickel (U Zürich)
\item    Geert Booij (Leiden U)
\item    Miriam Butt (U Konstanz)
\item    Ewa Dąbrowska (Northumbria U, Newcastle)
\item    Arnulf Deppermann (IDS, Mannheim)
\item    Nomi Erteschik-Shir (Ben Gurion U of the Negev)
\item    Martine Grice (U Cologne)
\item    Mutsumi Imai (Keio U at Shonan-Fujisawa)
\item    Laura Kallmeyer (U Düsseldorf)
\item    Manfred Krifka (Humboldt U Berlin and ZAS)
\item    Mary Esther Kropp Dakubu (U of Ghana)
\item    Aditi Lahiri (U Oxford)
\item    Stephen Levinson (MPI Psycholinguistics, Nijmegen)
\item    Anke Lüdeling (Humboldt U Berlin)
\item    Detmar Meurers (U Tübingen)
\item    Sam Mchombo (U of California, Berkeley)
\item    Rachel Nordlinger (U of Melbourne)
\item    Jairo Nunes (U São Paulo)
\item    Steven Pinker (Harvard U)
\item    Friedemann Pulvermüller (Freie U Berlin)
\item    Stuart Shieber (Harvard U)
\item    Dieter Stein (U Düsseldorf) 
% 24
\end{itemize}
\end{multicols}
}

\frame{
\frametitle{Reihen}

\begin{itemize}[<+->]
\item
African Language Grammars and Dictionaries\\
Adams Bodomo, Ken Hiraiwa, Firmin Ahoua

\item Comparative Studies on Niger-Congo Languages\\
Valentin Vydrin, Larry Hyman, Konstantin Pozdniakov, Guillaume Segerer, John Watters

\item
Studies in Diversity Linguistics\\
Martin Haspelmath, Fernando Zúñiga, Peter Arkadiev, Ruth Singer,
Pilar Valenzuela

\item
Empirically Oriented Theoretical Morphology and Syntax\\
Stefan Müller, Berthold Crysmann, Laura Kallmeyer 

\item
Implemented Grammars\\
Stefan Müller, Berthold Crysmann, Laura Kallmeyer 

\end{itemize}


}

\frame{
\frametitle{Reihen zweiter Teil}

\begin{itemize}[<+->]


\item
Lecture Notes in Language Sciences\\
Stefan Müller, Martin Haspelmath

\item Phonetics and Phonology\\
Martine Grice, Doris Mücke, Taehong Cho

(offizielle Reihe der Association for Laboratory Phonology, \url{labphon.org})

\item Topics at the Grammar-Discourse Interfaces\\
Philippa Cook, Anke Holler, Cathrine Fabricius-Hansen

\item Translation and Multilingual Natural Language Processing\\
Reinhard Rapp, Silvia Hansen-Schirra, Oliver Čulo

\end{itemize}
}

\frame{
\frametitle{Reihen under \\submission/""review}


\begin{itemize}[<+->]
\item Computational models of language evolution\\
      Luc Steels, Remi van Trijp

\item Language variation\\
      John Nerbonne, Remco Knooihuizen, Nanna Haug Hilton

\end{itemize}

}

\section{Statistik}

\frame{
\frametitle{Statistik}

\begin{itemize}[<+->]
\item In Vorbereitung: 19 
\item Eingereicht: 13 
\item Angenommen: 7 
\item Abgelehnt: 2 
\end{itemize}

}

\frame{
\frametitle{In der Endphase}

\vfill

% \includegraphics[width=.30\textwidth]{prosodic-detail}\hfill
% \includegraphics[width=.30\textwidth]{adjective-attribution}\hfill
% \includegraphics[width=.30\textwidth]{Alor-Pantar}

\vfill


}

\frame{
\frametitle{Fertig}

\centerline{
% \includegraphics[height=0.85\textheight]{typology-marked-s}
}



}

\section{Mitarbeiter}

\frame{
\frametitle{Mitarbeiter zur Zeit}

\begin{itemize}[<+->]
\item Projektorganisation/Press Manager:\\
      Stefan Müller und Martin Haspelmath
\item Projektkoordination: Sebastian Nordhoff (seit 25.02.2014)
\item Design: Dipl.-Des. Ulrike Harbort\\
      (Buch-Layout, Web-Konzept, Poster, Präsentationslayout)
\item Design-Beratung: Prof.\ Barbara Schmidt\\
      (Weißensee, Kunsthochschule Berlin)
\item Web-Blog: Frank Richter (\href{http://www.frank-m-richter.de/freescienceblog/}{Free Science Blog})
\item Web-Design: Robert Forkel (CSS)
\item \latex: Timm Lichte, Berthold Crysmann, Stefan Müller
\item Word Template: Antonio Machicao y Priemer
\item Proofreader: Eitan Grossman, Daniel W. Hieber, Aaron Sonnenschein
\end{itemize}

}


\section{Das Projekt}


\frame[shrink=15]{
\frametitle{Projektbestandteile}

\begin{itemize}[<+->]
\item Schaffung von Standards für \latex-Templates in der Linguistik
\item Automatische Konversion von \latex in diverse andere Formate\\
      (XML, e-books)
\item Schaffung einer Literaturdatenbank für Linguistik und Aufnahme entsprechender Einträge in \url{http://glottolog.org/}
\item Enhanced Publication
      \begin{itemize}
      \item ausklappbare Syntaxbäume (\url{http://hpsg.fu-berlin.de/OALI/rec2/rec2.html})
      \item Speicherung von Korpusdaten
      \item Sound-Dateien
      \item \ldots
      \end{itemize}
\item Erweiterungen von Open Monograph Press
      \begin{itemize}
      \item Anpassung an die deutschen Besonderheiten (VG-Wort, Kataloge, \ldots)
      \item Open Review (optional)\nocite{Pullum84a}
      \item Diskussionsphasen vor endgültiger Veröffentlichung (optional)
      \item Gamification (optional)
      \end{itemize}
\item alle Ergebnisse Open Source und für andere Disziplinen/Projekte zur Nachnutzung
\end{itemize}

\nocite{MH2013a}

}

\frame{
\frametitle{Stellen}


\begin{itemize}[<+->]
\item 1 Stelle: \latex/XML
\item 1 Stelle: Open Monograph Press-Erweiterungen
\item 1/2 Stelle: Support/Betreuung von Herausgebern/Autoren am CeDiS
\item 1/2 Stelle: BWLer: Dokumentation der Kosten, Geschäftsmodell\\
      (optionales Author Pays, optionale Bibliotheksbeiträge, Micropayment-Spenden, \ldots)
\bigskip
\item Ausschreibung in den nächsten Tagen
\end{itemize}

}

\section{Ausblick}

\frame{
\frametitle{Wie geht's weiter?}

\begin{itemize}
\item Farbe bekennen: \url{http://langsci-press.org/Meta/sign/}

(unser kleines \url{thecostofknowledge.com})

\item Mit anderen reden

\item Buch schreiben

\item Sich darüber freuen,\\ 
      dass das Buch auf der ganzen Welt gelesen werden kann.

\end{itemize}

}
\nocite{Shieber2012a}
\makeatletter
\setcounter{lastpagemainpart}{\the\c@framenumber}
\makeatother


\appendix

\section{Preise}

\frame{
\frametitle{Preisliste für \\Open-Access-Bücher}

\begin{itemize}
\item Palgrave Macmillan: 11,000 £ (17,500 \$)\footnote{
\url{http://poynder.blogspot.co.uk/2013/09/de-gruyters-sven-fund-on-state-of-open.html}. 04.03.2014.
}
\item Springer: ca.\, 15,000 €\footnote{
\url{http://poynder.blogspot.co.uk/2013/09/de-gruyters-sven-fund-on-state-of-open.html}. 04.03.2014.
\newline
Lustigerwiese steht das nicht in der FAQ: \url{http://www.springeropen.com/about/faq/chargesbooks}. 04.03.2014.
}
\item De Gruyter Open: 10.000 € (p.\,M.\ 02/2014)\\
      (1.500€ bei normalen Büchern, wenn nicht selbst gesetzt)
\item Brill: 5.000 €
\item Language Science Press: 0 €\\
      (Spenden und Publikationsmittel aus Projektgeldern willkommen)
\end{itemize}

}

\mode<beamer>{







\frame{
\frametitle{Brand}

\begin{itemize}[<+->]
\item Publishers live from their reputation.
\item For instance Elsevier charges \$5,000 for a paper in \emph{Cell} in an author pays Open Access modell.
\item Accaptence rate is 4\,\% = high prestige.
\item Scientists help building such brands.
\item So, why not working for our own benefit?
\end{itemize}


}


\section{Journals and Publishers}

\frame{
\frametitle{Linguistics Journals by Elsevier}

\begin{itemize}
\item 

    Assessing Writing, Computers and Composition, Discourse, Context and Media, 
    English for Specific Purposes, 
    Journal of Communication Disorders,
    Journal of English for Academic Purposes,
    Journal of Fluency Disorders,
    Journal of Neurolinguistics, 
    Journal of Phonetics, 
    Journal of Pragmatics, 
    Journal of Second Language Writing, 
    Language and Communication, 
    Language Sciences, 
    \alert{Lingua}
\end{itemize}
}

\frame{
\frametitle{Linguistics Journals by Springer}

\begin{itemize}
\item Linguistics and Philosophy,
Journal of Logic, Language and Information,
The Journal of Comparative Germanic Linguistics,
Journal of East Asian Linguistics,
Journal of Psycholinguistic Research,
Language Resources and Evaluation,
Machine Translation,
Morphology,
\alert{Natural Language \& Linguistic Theory},
An International Journal of Semantics and Its Interfaces in Grammar,
An International Journal of Modern and Mediaeval Language and Literature,
Russian Linguistics,
International Journal for the Study of Russian and other Slavic Languages
\end{itemize}

}

\frame{
\frametitle{Linguistics Journals by De Gruyter}

\begin{itemize}
\item Journal of African Languages and Linguistics, 
Linguistics,
\alert{Theoretical Linguistics},
\alert{The Linguistic Review},
Folia Linguistica,
Probus,
\alert{Zeitschrift für Sprachwissenschaft},
International Review of Applied Linguistics in Language Teaching,
Zeitschrift für Angewandte Linguistik,
Journal of Politeness Research,
\alert{Cognitive Linguistics},
Chinese as a Second Language Research,
Journal of Literary Semantics,
Multicultural Learning and Teaching,
Intercultural Pragmatics,
Zeitschrift für Rezensionen zur germanistischen Sprachwissenschaft,
Humor,
Text \& Talk,
Beiträge zur Geschichte der deutschen Sprache und Literatur,
Germanistik,
Anglia

\end{itemize}

}

}

\nocite{MuellerOA}

% muß immer geladen werden, wegen Referenzen
% \input{lsp-literatur}

\setcounter{framenumber}{\thelastpagemainpart}



\end{document}